\documentclass[10pt, a4paper, oneside]{article}

\usepackage[utf8]{inputenc}
\usepackage{fancyhdr}
\usepackage{geometry}
\usepackage{enumerate}
\usepackage{amsthm}
\usepackage{amsmath}
\usepackage{mathtools}
\usepackage{amssymb}
\usepackage{latexsym}
\usepackage[round]{natbib}
\usepackage[final]{pdfpages}

\bibpunct{[}{]}{;}{a}{}{,}

\newtheorem{thm}{Theorem}[section]
\theoremstyle{definition}
\newtheorem{dfn}[thm]{Definition}
\theoremstyle{remark}
\newtheorem{note}[thm]{Note}
\theoremstyle{plain}
\newtheorem{lem}[thm]{Lemma}
\theoremstyle{plain}
\newtheorem{cor}[thm]{Corollary}

\newcommand{\defspace}[1]{\hspace{#1} &=_{df} \hspace{#1}}

\title{Formal Methods Seminar 2012:\\
       The compactness and Löwenheim--Skolem theorems}
\author{Benedict Eastaugh\footnote{Much of this material was adapted from
        Wilfrid Hodges's excellent textbook, \emph{A Shorter Model Theory}.
        Special thanks are due to Kate Hodesdon and Paul Ross for their
        invaluable input and feedback.}}
\date{February 9th, 2012}

\begin{document}

\maketitle

First-order logic is a powerful and expressive system which has been used to
formalise many basic systems of mathematics, including set theory, arithmetic,
and the real closed field. It has been intensively studied since the early 20th
century. Here we present two of the most important metalogical results about
first-order logic: the compactness and Löwenheim--Skolem theorems.

As well as defining the notions required to state and prove these theorems, we
shall examine some of their philosophical impact, which has made itself felt not
only in logic and philosophy of mathematics but also in the philosophy of
language.


\section{The compactness theorem}

The compactness theorem was originally proved by \citet{godel1930} as a
corollary to his completeness theorem for first-order logic.

\begin{lem}
    \label{completeness}
    (Completeness)
    If $S$ is a consistent set of sentences then $S$ has a model.
\end{lem}

\begin{thm}
    \label{compactness}
    (Compactness)
    Let $S$ be a set of wffs of first-order logic. If every finite subset of $S$
    is satisfiable, then $S$ is satisfiable.
\end{thm}

\begin{proof}
    Consider the contraposition of compactness: if $S$ is not satisfiable, then
    there is a finite subset of $S$ which is not satisfiable. So assume that $S$
    is not satisfiable: there is no structure $A$ such that $A \models S$. We
    thus merely need to prove that there exists a finite subset $S' \subseteq S$
    such that $S'$ is not satisfiable.
    
    By lemma \ref{completeness}, if $S$ is consistent then $S$ has a model. So
    if $S$ has no model then $S$ is inconsistent. For a theory to be
    inconsistent is just for it to prove a contradiction---that is, there is
    some wff $\psi$ such that $S \vdash \psi \wedge \neg\psi$. Consider a proof
    witnessing this statement,\footnote{The term `witness' means an object
    satisfying an existential assertion.}
    
    \begin{displaymath}
        \Theta = \langle \theta_1, \dotsc, \theta_n \rangle
    \end{displaymath}
    
    for $n \in \mathbb{N}$. Since $\Theta$ must be finitely long, only finitely
    many $\theta_i$ can appear in it such that $\theta_i \in S$. Let $S'$ be the
    set of wffs of $S$ which appear in $\Theta$. $S'$ must also be finite, and
    of course $S' \vdash \psi \wedge \neg\psi$, so $S'$ is inconsistent, and
    thus is a finite subset of $S$ which is not satisfiable.
\end{proof}

There are also purely model-theoretic proofs of the compactness theorem---see
for example \citealt[pp. 125--127]{hodges1997}.

\subsection{Applications of compactness}

The compactness theorem is a model-existence theorem: it says that \emph{given}
that some condition holds, \emph{there exists a model} with certain properties.
A simple application of the theorem is showing that given the consistency of
some theory, non-standard models of that theory exists. This is easily seen in
the case of arithmetic.

\begin{lem}
    There exists a model of first-order Peano Arithmetic, PA, which is not
    isomorphic to the standard model.
\end{lem}

\begin{proof}
    Add a new constant symbol $e$ to the language of arithmetic and for each
    numeral $\bar{n}$ add the sentence $\bar{n} < e$ to the axioms of PA to
    obtain the new theory $S$. Any finite $T \subseteq S$ is satisfied by the
    usual natural number structure $\mathbb{N}$ by interpreting the constant $e$
    as some number larger than any numeral employed in the sentences of $T$. So
    by compactness, $S$ has a model $M$ which contains an element $e^M$ greater
    than any standard natural number. As PA is a subtheory of $S$, $M$ is also a
    model of PA.
\end{proof}

Compactness is also useful for proving that certain classes of structures are
finitely axiomatisable; for details see \citealt[pp. 114--116]{vdalen2004}.
However, the best-known application of compactness is in proving the upward
direction of the \emph{Löwenheim--Skolem theorem}.


\section{The Löwenheim--Skolem theorem}

From the compactness theorem we first learn about non-standard models:
structures which satisfy a first-order theory $S$ and yet are not isomorphic to
the intended interpretation. Non-standard models of arithmetic bring this out
quite forcefully: these strange structures are nothing like we naïvely expect
from a theory which seems, in its essentials, so transparent.

One aspect of this is cardinality: we think of arithmetic as inherently
describing a countable structure, and indeed our very notion of a set being
countable is that it can be put into a one-to-one correspondence with the
natural numbers. But not only are there uncountable models of first-order
theories such as Peano Arithmetic, but there are arbitrarily large models of
every first-order theory.

Perhaps even more surprisingly, no matter how large the intended model of a
first-order theory is, it also has a countable model. This result was originally
proved in 1915 by the German mathematician Leopold Löwenheim. A simpler and more
general proof was given in 1920 by the Norwegian logician Thoralf Skolem
\citep{lowenheim1915, skolem1920}.

\subsection{Some model theory}

In order to state the theorem in its full generality we need a few definitions
from model theory, which allow us to characterise languages and structures in a
more fine-grained way. By convention for a structure $A$ we add a superscript to
denote an element, function or relation belonging to that structure. For
instance, given a constant $c$ we would denote its referant in $A$ by $c^A$.

\subsubsection{Languages, names and theories}

The languages we have discussed so far have all been \emph{countable}---that is,
their formulae can be enumerated by assigning a unique natural number to each
formula. However, it is often convenient to be able to add new non-logical
symbols to a language---perhaps even uncountably many of them. Under some
circumstances we thus need to pay heed to the \emph{cardinality} of a language:
roughly speaking, how many different formulae it includes.

\begin{dfn}
    Two formulae are \emph{variants} of one another if they differ only in their
    choice of variable names---that is, if one can be obtained from the other by
    a uniform substitution of variables. $\varphi(x_1, \dotsc, x_n)$ is a
    variant of $\varphi(y_1, \dotsc, y_n)$, and vice versa, while
    $\forall{x} (P(x))$ is a variant of $\forall{y} (P(y))$.
\end{dfn}

\begin{dfn}
    The \emph{cardinality} of a first-order language $L$, $|L|$, is the number
    of equivalence classes of formulae of $L$ under the relation of being
    variants.
\end{dfn}

\begin{lem}
    Let $L$ be a first-order langauge and let $\kappa$ be the number of
    non-logical symbols of $L$. Then $|L| = \omega + \kappa =
    max \left\{ \omega, \kappa \right\}$.
\end{lem}

\begin{proof}
    Exercise.
\end{proof}

\begin{dfn}
    Let $L$ be a first-order language and $A$ an $L$-structure. Then the
    \emph{theory of $A$}, $Th(A)$, is the set of all sentences $\varphi$ of $L$
    such that $A \models \varphi$.
\end{dfn}

\begin{dfn}
    Suppose we have two signatures, $L^-$ and $L^+$, and that $L^- \subseteq
    L^+$. Then if $A$ is an $L^+$-structure we can transform it into an
    $L^-$-structure by forgetting the symbols of $L^+$ which don't appear in
    $L^-$. This gives us the \emph{$L^-$-reduct} of $A$, $A| L^-$.
\end{dfn}

\subsubsection{Extensions and elementarity}

We now look at the relations structures can bear to one another. These notions
from model theory are common currency in logic and philosophy of mathematics,
and worth learning for their own sake, but here we use them to prove stronger
versions of our goal theorems.

We denote $n$-tuples of constants, variables and elements $\langle a_1, \dotsc,
a_n \rangle$ as $\bar{c}$, $\bar{x}$, $\bar{a}$ and so on. In the case of a
function applied to a tuple, $f\bar{a}$, we mean $\langle f(a_1), \dotsc, f(a_n)
\rangle$.

\begin{dfn}
    Suppose $A$ and $B$ are $L$-structures. An \emph{embedding} $f$ from $A$
    into $B$, $f : A \rightarrow B$, is a function $f$ from $dom(A)$ to $dom(B)$
    obeying the following conditions.
    
    \begin{enumerate}
        \item $f$ is injective.
        \item For each constant $c$ of $L$, $f(c^A) = c^B$.
        \item For each $n > 0$ and each $n$-ary relation symbol $R$ of $L$ and
              $n$-tuple $\bar{a}$ of elements of $A$, $\bar{a} \in R^A
              \leftrightarrow f\bar{a} \in R^B$.
        \item For each $n > 0$ and each $n$-ary function symbol $F$ of $L$ and
              $n$-tuple $\bar{a}$ of elements of $A$, $f(F^A(\bar{a})) =
              F(f\bar{a})$.
    \end{enumerate}
\end{dfn}

\begin{dfn}
    Given two $L$-structures $A$ and $B$ such that $dom(A) \subseteq dom(B)$, if
    the inclusion map $i : dom(A) \rightarrow dom(B)$ is an embedding then we
    say that $B$ is an \emph{extension} of $A$, or alternatively that $A$ is a
    \emph{substructure} of $B$, $A \subseteq B$.
\end{dfn}

\begin{dfn}
    If an embedding between $L$-structures $f : A \rightarrow B$ preserves
    first-order formulae, $A \models \varphi(\bar{a}) \rightarrow B \models
    \varphi(f\bar{a})$, then we say that $f$ is an \emph{elementary embedding}.
    
    Suppose that $B$ is an extension of $A$. If the inclusion map is an
    elementary embedding then $B$ is an \emph{elementary extension} of $A$, or
    equivalently, $A$ is an \emph{elementary substructure} of $B$, $A
    \preccurlyeq B$.
\end{dfn}

\begin{dfn}
    Let $A$ be an $L$-structure and $\bar{a}$ be a sequence of all elements of
    $A$. We choose a sequence $\bar{c}$ of new constants not in $L$ to name the
    elements of $\bar{a}$. Adjoining these constants to $L$ gives us a new
    language $L(\bar{c})$ and an $L(\bar{c})$-structure $(A, \bar{a})$ which is
    just $A$ with the constants $\bar{c}$ interpreted as the constant elements
    $\bar{a}$.
    
    Then the \emph{elementary diagram of $A$}, $eldiag(A)$, is $Th(A, \bar{a})$.
    In other words, it's the set of sentences \emph{with parameters from $A$}
    that are true in $A$.
\end{dfn}

\begin{lem}
    \label{eldiag_lem}
    Let $L$ be a first-order language and $A$ an $L$-structure. If $B \models
    eldiag(A)$, then there is an elementary embedding of $A$ into the reduct
    $B | L$.
\end{lem}

\begin{proof}
    See \citealt[p. 49]{hodges1997}.
\end{proof}

\subsection{The upwards Löwenheim--Skolem theorem}

One consequence of compactness is what is often called the \emph{upwards
Löwenheim--Skolem theorem}: that given any theory $S$ with an infinite model
$A$, we can always expand the model to one of a higher cardinality. The name is
somewhat inappropriate, since as we shall see, Skolem rejected the existence of
uncountable sets.

\begin{thm}
    \label{up_lst}
    (Upwards Löwenheim--Skolem theorem)
    Let $L$ be a first-order language, $A$ an infinite $L$-structure and
    $\kappa$ a cardinal such that $|L| \leq \kappa$ and $|A| \leq \kappa$. Then
    $A$ has an elementary extension of cardinality $\kappa$.
\end{thm}

\begin{proof}
    Providing names for the elements of $A$ as necessary, let $T = eldiag(A)$.
    Add new constants $c_i$ to $L$ for $i < \kappa$, obtaining the language
    $L^*$. Let $S$ be the theory
    
    \begin{displaymath}
        T \cup \left\{\ c_i \neq c_j\ |\ i, j < \kappa\ \right\}.
    \end{displaymath}
    
    Pick an arbitrary finite $U \subseteq S$. Only finitely many of the new
    constants $c_i$ occur in the sentences of $U$, and $A$ is infinite so we
    simply assign each of the new constants to an element $c^A_i$ of $A$.
    Clearly $A \models U$ so by compactness there is an $L^*$-structure $B$
    satisfying $S$.
    
    $B \models eldiag(A)$, so by lemma \ref{eldiag_lem} there is an elementary
    embedding $e : A \rightarrow B | L$. Elementary embeddings are injections,
    so we can straightforwardly replace $e(x^A) = x^B$ in $B | L$ with $x^A$,
    thus making $B | L$ an elementary extension of $A$.
    
    Since $B \models S$, for every pair of elements $c^B_i$ and $c^B_j$ with
    $i, j < \kappa$, $c^B_i \neq c^B_j$. The cardinality of $B | L$ is therefore
    at least $\kappa$. We can apply theorem \ref{down_lst} to reduce the
    cardinality of $B | L$ to exactly $\kappa$.
\end{proof}

\subsection{The downwards Löwenheim--Skolem theorem}

\begin{thm}
    \label{down_lst}
    (Downwards Löwenheim--Skolem theorem)
    Let $L$ be a first-order language, $A$ an $L$-structure and $\kappa$ an
    infinite cardinal such that $|L| \leq \kappa \leq |A|$. Then $A$ has an
    elementary substructure of cardinality $\kappa$.
\end{thm}

\begin{proof}
    Omitted, but see \citealt[pp. 69--72]{hodges1997}.
\end{proof}

Combining the upward and downward parts of the theorem, we can see that in
general, if a first-order theory $S$ has at least one infinite model, then it
has models of every infinite cardinality. This result demonstrates that
first-order logic is too weak to distinguish between different infinite
cardinalities.

If we take the cumulative hierarchy picture at face value then the axioms of ZFC
appear to describe a structure containing many infinite ordinals and cardinals.
The Löwenheim--Skolem theorem demonstrates that there are many models of ZFC in
which the higher reaches of the transfinite model do not appear. There are even
\emph{countable} models of ZFC. This seems to fly in the face of the
assertion---a theorem of ZFC---that an uncountable set exists. This phenomenon
is known as \emph{Skolem's paradox}.

\subsection{Skolem's paradox}

Skolem was a sceptic about the existence of uncountable sets, and he took the
Löwenheim--Skolem theorem as evidence that his scepticism was justified and
that set-theoretic concepts were inherently \emph{relative}.

Let's examine one instance of this conceptual relativity, by looking more
closely at cardinality. For a set $Y$ to have a greater cardinality than another
set $X$ is just for there to be no surjective function $f : X \rightarrow Y$. So
it is easy to see how $|X| < |Y|$ could be true in a model $\mathcal{M}$ of ZFC,
just because the model didn't include the necessary surjection, even if there
were a larger structure $\mathcal{M'}$ which did include it.

Skolem contended that to obtain something absolutely uncountable we would either
have to start with uncountably many axioms or with an axiom that could yield
uncountably many first-order propositions (in other words, higher-order
quantification). However, to take either route would be begging the question,
and assuming the existence of uncountable objects in order to prove it.

\citet{jane2001} is a recent article examining Skolem's view of the relativity
of set-theoretic concepts, while \citet{bays2007} is a thorough interrogation of
the mathematics of the paradox.

\subsection{Putnam's model-theoretic argument}

Hilary Putnam \citeyearpar{putnam1980} famously put Skolem's paradox to work in
the philosophy of language, arguing that the relativity of set-theoretic
concepts is also a problem for concepts expressed in natural language, and for
scientific theories. Because no first-order theory with an infinite model can
determine its interpretation up to isomorphism, the theoretical constraints
given by the theory cannot determine reference. Neither can operational
constraints since these are ``just more theory''.

For Putnam, this indeterminacy of interpretation \emph{only} arises given a bad
way of thinking about theories. Taking truth to be truth in an intended model
and taking intended models to be those satisfying sentences which we want to
come out true will always give us too many models. But stipulating sentences to
be satisfied is the wrong way to go about fnding intended models, since in order
to obtain the sentences that comprise our ideal theory we must already have an
interpretation of the language they're expressed in. Thus, even generating
Skolemite problems for our own language requires fixity of reference.

These considerations led Putnam to reject metaphysical realism and endorse a
Dummetian, anti-realist semantics.

\begin{quote}
    [T]he world does not pick models or interpret languages. \emph{We} interpret
    our languages or nothing does.
    
    We need, therefore, a standpoint which links use and reference in just the
    way that the metaphysical realist standpoint refuses to do. The standpoint
    of ``non-realist semantics'' is precisely that standpoint.
    \citep[p. 482]{putnam1980}
\end{quote}


\section{Lindström's theorem}

Compactness and the downward Löwenheim--Skolem theorem demonstrate two
important properties of first-order logic. The Swedish logician Per Lindström
proved in the 1960s that first-order logic is maximal with respect to these
properties: it is the strongest logic which has both the compactness property
and the downwards Löwenheim--Skolem property.\footnote{For a proof of the
theorem, see \citet{vaananen2010}, which is available online.}

Put another way, any logic which goes beyond first-order must distinguish
between infinite cardinalities: there must be sentences that have models of
some infinite cardinalities but not others, and thus make non-trivial assertions
about our background ontology, usually taken to be the cumulative hierarchy of
sets.

The obvious example of such a logic is second-order logic, which allows us to
provide a categorical axiomatisation of the natural number structure, ruling out
models of cardinality greater than $\omega$, and also a quasi-categorical
axiomatisation of set theory where the only set-sized models are proper initial
segments of $V$ satisfying certain closure constraints.

However, second-order logic has significant technical disadvantages compared to
first-order logic. In particular, it does not admit a complete proof theory.
Many philosophers have also shied away from the additional conceptual and
ontological commitments thought to be entailed by quantification over
properties, although in recent decades this has changed somewhat and
second-order logic has become more acceptable.

\bibliographystyle{abbrvnat}
\bibliography{metalogic-fm2012}

\end{document}
