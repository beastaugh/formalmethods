\documentclass[10pt, a4paper, oneside]{article}

\usepackage[utf8]{inputenc}
\usepackage{fancyhdr}
\usepackage{geometry}
\usepackage{enumerate}
\usepackage{amsthm}
\usepackage{amsmath}
\usepackage{mathtools}
\usepackage{amssymb}
\usepackage{latexsym}
\usepackage[round]{natbib}
\usepackage[final]{pdfpages}

\bibpunct{[}{]}{;}{a}{}{,}

\newtheorem{thm}{Theorem}[section]
\theoremstyle{definition}
\newtheorem{dfn}[thm]{Definition}
\theoremstyle{remark}
\newtheorem{note}[thm]{Note}
\theoremstyle{plain}
\newtheorem{lem}[thm]{Lemma}

\title{Formal Methods Seminar 2012:\\
       The completeness, compactness and\\
       Löwenheim--Skolem theorems}
\author{Benedict Eastaugh and Kate Hodesdon}
\date{February 2nd and 9th, 2012}

\begin{document}

\maketitle

First-order logic is a powerful and expressive system which has been used to
formalise many basic systems of mathematics, including set theory, arithmetic,
and the real closed field. It has been intensively studied since the early 20th
century. The most basic and important results are the completeness, compactness
and Löwenheim--Skolem theorems.


\section{The language}


\section{The semantics}


\section{Proofs}


\section{Soundness and completeness}

\subsection{Soundness}

\subsection{Completeness}


\section{Compactness}

Another model-existence theorem is the \emph{compactness theorem}.


Although the compactness theorem is easily derived from the completeness
theorem, there are also purely model-theoretic proofs of the theorem.

\section{The Löwenheim--Skolem theorem}

\subsection{The upwards Löwenheim--Skolem theorem}

One easy consequence of compactness is what is often called the \emph{upwards
Löwenheim--Skolem theorem}: that given any theory $S$ with an infinite model
$A$, we can always expand the model to one of a higher cardinality.

\begin{thm}
    (Upwards Löwenheim--Skolem theorem)
    
    Let $L$ be a first-order language with cardinality $\leq \lambda$, and let $A$
    be an infinite $L$-structure with cardinality $\leq \lambda$. Then there
    exists an $L$-structure $B$ of cardinality $\lambda$ such that $B$ is an
    elementary extension of $A$.
\end{thm}

\subsection{The downwards Löwenheim--Skolem theorem}

\begin{thm}
    (Downwards Löwenheim--Skolem theorem)
    
    Let $L$ be a first-order language and let $A$ be an $L$-structure with
    $X \subseteq dom(A)$. Given some cardinal $\lambda$ such that
    $|L| + |X| \leq \lambda \leq |A|$, there exists an elementary substructure
    $B$ of $A$ with $|B| = \lambda$ and $X \subseteq dom(B)$.
\end{thm}


\bibliographystyle{abbrvnat}
\bibliography{metalogic-fm2012}

\end{document}
