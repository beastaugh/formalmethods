\documentclass[10pt, a4paper, oneside]{article}

\usepackage[utf8]{inputenc}
\usepackage{fancyhdr}
\usepackage{geometry}
\usepackage{enumerate}
\usepackage{amsthm}
\usepackage{amsmath}
\usepackage{mathtools}
\usepackage{amssymb}
\usepackage{latexsym}
\usepackage[round]{natbib}
\usepackage[final]{pdfpages}

\bibpunct{[}{]}{;}{a}{}{,}

\newtheorem{thm}{Theorem}[section]
\theoremstyle{definition}
\newtheorem{dfn}[thm]{Definition}
\theoremstyle{remark}
\newtheorem{note}[thm]{Note}
\theoremstyle{plain}
\newtheorem{lem}[thm]{Lemma}

\newcommand{\defspace}[1]{\hspace{#1} &=_{df} \hspace{#1}}

\title{Formal Methods Seminar 2012:\\
       The compactness and Löwenheim--Skolem theorems}
\author{Benedict Eastaugh}
\date{February 9th, 2012}

\begin{document}

\maketitle

First-order logic is a powerful and expressive system which has been used to
formalise many basic systems of mathematics, including set theory, arithmetic,
and the real closed field. It has been intensively studied since the early 20th
century. Here we present two of the most important metalogical results about
first-order logic: the compactness and Löwenheim--Skolem theorems. Together
they


\section{The compactness theorem}

\begin{thm}
    (Compactness) Let $S$ be a set of wffs of first-order logic. If every finite
    subset of $S$ is satisfiable, then $S$ is satisfiable.
\end{thm}

A proof of the compactness theorem follows fairly directly from the
\emph{completeness theorem} for first-order logic, that given some set $S$ of
wffs of first-order logic and a wff $\varphi$, if $S \models \varphi$ then
$S \vdash \varphi$.

\begin{proof}
    Consider the contraposition of compactness: if $S$ is not satisfiable, then
    there is a finite subset of $S$ which is not satisfiable. So assume that $S$
    is not satisfiable: there is no model $\mathcal{M}$ such that $\mathcal{M}
    \models S$. We thus merely need to prove that there exists a finite subset
    $S' \subseteq S$ such that $S'$ is not satisfiable.
    
    Recall the following reformulation of the completeness theorem: if $S$ is
    consistent, then $S$ has a model. So if $S$ has no model then $S$ is
    inconsistent. A set of wffs $S$ is inconsistent just in case there is some
    wff $\psi$ such that $S \vdash \psi \wedge \neg\psi$. So consider a proof
    which witnesses this statement,
    
    \begin{displaymath}
        \Theta = \langle \theta_1, \dotsc, \theta_n \rangle
    \end{displaymath}
    
    for $n \in \mathbb{N}$. Since $\Theta$ must be finitely long, only finitely
    many $\theta_i$ can appear in it such that $\theta_i \in S$. So let $S'$ be
    the set of wffs of $S$ which appear in $\Theta$. $S'$ must also be finite,
    and of course $S' \vdash \psi \wedge \neg\psi$, so $S'$ is inconsistent, and
    thus is a finite subset of $S$ which is not satisfiable.
\end{proof}

There are also purely model-theoretic proofs of the compactness theorem---see
for example \citet[pp. 125--127]{hodges1997}.


\section{The Löwenheim--Skolem theorem}

From the compactness theorem we first learn about non-standard models:
structures which satisfy a first-order theory $S$ and yet are not isomorphic to
the intended interpretation. Non-standard models of arithmetic bring this out
quite forcefully: these strange structures are nothing like we naïvely expect
from a theory which seems, in its essentials, so transparent.

The Löwenheim--Skolem theorem gives us a sense of quite how widespread this
phenomenon is. Not only are there uncountable models of theories we think of as
describing countable structures like the natural numbers, but there are even
countable models of theories whose intended models are uncountable.

This is perhaps most striking in the case of set theory. From the cumulative
hierarchy picture of the set-theoretic universe, we would expect that any model
of set theory would contain many transfinite cardinalities. The
Löwenheim--Skolem theorem induces what Skolem called the ``relativity of set
theory'': there are models of set theory which

The languages we have discussed so far have all been \emph{countable}---that is,
their formulae can be enumerated by assigning a unique natural number to each
formula. However, it is often convenient to be able to add new non-logical
symbols to a language---perhaps even uncountably many of them. Under some
circumstances we thus need to pay heed to the \emph{cardinality} of a language:
roughly speaking, how many different formulae it includes.

\begin{dfn}
    Two formulae are \emph{variants} of one another if they differ only in their
    choice of variable names---that is, if one can be obtained from the other by
    a uniform substitution of variables. $\varphi(x_1, \dotsc, x_n)$ is a
    variant of $\varphi(y_1, \dotsc, y_n)$, and vice versa, while
    $\forall{x} (P(x))$ is a variant of $\forall{y} (P(y))$.
    
    The \emph{cardinality} of a first-order language $L$, $|L|$, is the number
    of equivalence classes of formulae of $L$ under the relation of being
    variants. It's easy to see that the cardinality of the first-order language
    with the empty signature (that is to say, containing no non-logical symbols)
    is $\aleph_0$.
\end{dfn}

\begin{dfn}
    Elementary extensions and substructures.
\end{dfn}

\subsection{The upwards Löwenheim--Skolem theorem}

One easy consequence of compactness is what is often called the \emph{upwards
Löwenheim--Skolem theorem}: that given any theory $S$ with an infinite model
$A$, we can always expand the model to one of a higher cardinality.

\begin{thm}
    (Upwards Löwenheim--Skolem theorem) Let $L$ be a first-order language with
    cardinality $\leq \lambda$, and let $A$ be an infinite $L$-structure with
    cardinality $\leq \lambda$. Then there exists an $L$-structure $B$ of
    cardinality $\lambda$ such that $B$ is an elementary extension of $A$.
\end{thm}

\subsection{The downwards Löwenheim--Skolem theorem}

\begin{thm}
    (Downwards Löwenheim--Skolem theorem) Let $L$ be a first-order language and
    let $A$ be an $L$-structure with $X \subseteq dom(A)$. Given some cardinal
    $\lambda$ such that $|L| + |X| \leq \lambda \leq |A|$, there exists an
    elementary substructure $B$ of $A$ with $|B| = \lambda$ and
    $X \subseteq dom(B)$.
\end{thm}


\bibliographystyle{abbrvnat}
\bibliography{metalogic-fm2012}

\end{document}
